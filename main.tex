\documentclass[12pt]{article}
\usepackage[portuguese]{babel}
\usepackage{amsmath}
\newcommand{\numpy}{{\tt numpy}}   

\topmargin -.5in
\textheight 9in
\oddsidemargin -.25in
\evensidemargin -.25in
\textwidth 7in

\begin{document}

\author{Juan Carlos Teran}
\title{Mecânica Quântica I: Lista 3: Spin 1/2}
\maketitle

\medskip

\begin{enumerate}

\item
(a) Utilizando as representações para as componentes de um spin 1/2 em termos das matrizes de Pauli, verifique a relação de comutação:
\begin{equation}
  [\hat{S}_x, \hat{S}_y] = i\hbar \hat{S}_z  
\end{equation}

As componentes do spin \(\hat{S}_x\), \(\hat{S}_y\) e \(\hat{S}_z\) para um sistema de spin \(1/2\) são dadas pelas matrizes de Pauli, multiplicadas por \(\frac{\hbar}{2}\):

\[
\hat{S}_x = \frac{\hbar}{2} \sigma_x, \quad \hat{S}_y = \frac{\hbar}{2} \sigma_y, \quad \hat{S}_z = \frac{\hbar}{2} \sigma_z,
\]

onde \(\sigma_x\), \(\sigma_y\) e \(\sigma_z\) são as matrizes de Pauli:

\[
\sigma_x = \begin{pmatrix}
0 & 1 \\
1 & 0
\end{pmatrix}, \quad
\sigma_y = \begin{pmatrix}
0 & -i \\
i & 0
\end{pmatrix}, \quad
\sigma_z = \begin{pmatrix}
1 & 0 \\
0 & -1
\end{pmatrix}.
\]

A relação de comutação entre as componentes do spin é dada por:

\[
[\hat{S}_x, \hat{S}_y] = \hat{S}_x \hat{S}_y - \hat{S}_y \hat{S}_x.
\]

Substituindo \(\hat{S}_x\) e \(\hat{S}_y\) pelas suas respectivas expressões:

\[
[\hat{S}_x, \hat{S}_y] = \left(\frac{\hbar}{2} \sigma_x\right)\left(\frac{\hbar}{2} \sigma_y\right) - \left(\frac{\hbar}{2} \sigma_y\right)\left(\frac{\hbar}{2} \sigma_x\right).
\]

\[
[\hat{S}_x, \hat{S}_y] = \frac{\hbar^2}{4} \left( \sigma_x \sigma_y - \sigma_y \sigma_x \right).
\]

Sabemos que:

\[
\sigma_x \sigma_y = i\sigma_z, \quad \sigma_y \sigma_x = -i\sigma_z.
\]

Portanto:

\[
\sigma_x \sigma_y - \sigma_y \sigma_x = 2i\sigma_z.
\]

Assim, a relação de comutação se torna:

\[
[\hat{S}_x, \hat{S}_y] = \frac{\hbar^2}{4} \cdot 2i\sigma_z = i\hbar \frac{\hbar}{2} \sigma_z = i\hbar \hat{S}_z.
\]

Portanto, verificamos que:

\[
[\hat{S}_x, \hat{S}_y] = i\hbar \hat{S}_z.
\]


(b) Utilizando este resultado junto com a desigualdade de Heisenberg geral, prove que, em qualquer estado $| \psi \rangle$ de um spin 1/2,

$\Delta S_x \cdot \Delta S_y \geq \frac{\hbar}{2} |\langle \hat{S}_z \rangle|$

Para provar a desigualdade de Heisenberg, usaremos a relação de incerteza geral:

\[
\Delta A \cdot \Delta B \geq \frac{1}{2} |\langle [A, B] \rangle|,
\]

onde \(A = \hat{S}_x\) e \(B = \hat{S}_y\). Substituindo a relação de comutação que acabamos de verificar:

\[
\Delta S_x \cdot \Delta S_y \geq \frac{1}{2} |\langle [\hat{S}_x, \hat{S}_y] \rangle|.
\]

Sabemos que:

\[
\langle [\hat{S}_x, \hat{S}_y] \rangle = i\hbar \langle \hat{S}_z \rangle.
\]

Portanto:

\[
\Delta S_x \cdot \Delta S_y \geq \frac{1}{2} |\langle i\hbar \hat{S}_z \rangle| = \frac{\hbar}{2} |\langle \hat{S}_z \rangle|.
\]

Assim, provamos que:

\[
\Delta S_x \cdot \Delta S_y \geq \frac{\hbar}{2} |\langle \hat{S}_z \rangle|.
\]

Essa é a desigualdade de Heisenberg para um sistema de spin \(1/2\).
\item
Podemos mostrar que para qualquer vetor $| \chi \rangle$ representando o estado de um spin 1/2, existe uma direção  $\hat{n}$ tal que $| \chi \rangle = | + \rangle_{\hat{n}}$, onde $| + \rangle_{\hat{n}}$ é o autovetor de $\hat{\vec{S}} \cdot \hat{n}$ com autovalor positivo:

(a) Obtenha os valores esperados $\langle \hat{S}_x \rangle, \langle \hat{S}_y \rangle$ e $\langle \hat{S}_z \rangle$, em função das coordenadas esféricas $\theta$, $\varphi$ do vetor unitário $\hat{n}$. Interprete os resultados.

Vamos resolver envolvendo um sistema de spin \(\frac{1}{2}\).

Valores Esperados \(\langle \hat{S}_x \rangle\), \(\langle \hat{S}_y \rangle\) e \(\langle \hat{S}_z \rangle\) 
Considere que o vetor unitário \(\hat{n}\) na direção \((\theta, \varphi)\) em coordenadas esféricas pode ser escrito como:
\begin{equation}
    \hat{n} = \sin\theta \cos\varphi \, \hat{x} + \sin\theta \sin\varphi \, \hat{y} + \cos\theta \, \hat{z}
\end{equation}

Podemos expressar o operador \(\hat{\vec{S}} \cdot \hat{n}\) como:

\begin{equation}
  \hat{\vec{S}} \cdot \hat{n} = \hat{S}_x \sin\theta \cos\varphi + \hat{S}_y \sin\theta \sin\varphi + \hat{S}_z \cos\theta.  
\end{equation}

O estado \(|+\rangle_{\hat{n}}\) é o autovetor de \(\hat{\vec{S}} \cdot \hat{n}\) com autovalor \(\frac{\hbar}{2}\). Como \(|\chi\rangle = |+\rangle_{\hat{n}}\), o valor esperado de \(\hat{\vec{S}} \cdot \hat{n}\) é:

\begin{equation}
 \langle \chi | \hat{\vec{S}} \cdot \hat{n} | \chi \rangle = \frac{\hbar}{2}   
\end{equation}

Para obter os valores esperados \(\langle \hat{S}_x \rangle\), \(\langle \hat{S}_y \rangle\) e \(\langle \hat{S}_z \rangle\), podemos igualar os coeficientes correspondentes:

\begin{equation}
\langle \hat{S}_x \rangle = \frac{\hbar}{2} \sin\theta \cos\varphi \quad \wedge \quad  
\langle \hat{S}_y \rangle = \frac{\hbar}{2} \sin\theta \sin\varphi \quad \wedge  \quad
\langle \hat{S}_z \rangle = \frac{\hbar}{2} \cos\theta.
\end{equation}

Esses valores esperados correspondem às projeções do momento angular de spin ao longo das direções \(x\), \(y\) e \(z\), respectivamente. Eles dependem diretamente da orientação do vetor \(\hat{n}\) em coordenadas esféricas, o que reflete a direção em que o spin está medido.

(b) Obtenha as incertezas quadradas $\Delta S_x \cdot \Delta S_y \geq \frac{\hbar}{2} |\langle S_z \rangle|$ em função das coordenadas esféricas $\theta$, $\varphi$ do vetor unitário $\hat{n}$.

Para estados \(|\chi\rangle = |+\rangle_{\hat{n}}\), os valores esperados \(\langle \hat{S}_x \rangle\), \(\langle \hat{S}_y \rangle\), e \(\langle \hat{S}_z \rangle\) já foram obtidos. Agora, precisamos encontrar as incertezas \( \Delta S_x \) e \( \Delta S_y \).

Como \(|\chi\rangle\) é um autovetor de \(\hat{\vec{S}} \cdot \hat{n}\) com autovalor \(\frac{\hbar}{2}\):

\[
\langle \hat{S}_x^2 \rangle = \frac{\hbar^2}{4}, \quad \langle \hat{S}_y^2 \rangle = \frac{\hbar^2}{4}, \quad \langle \hat{S}_z^2 \rangle = \frac{\hbar^2}{4}.
\]

Portanto, as incertezas são:

\[
\Delta S_x = \sqrt{\langle \hat{S}_x^2 \rangle - \langle \hat{S}_x \rangle^2} = \frac{\hbar}{2} \sqrt{1 - \sin^2\theta \cos^2\varphi},
\]
\[
\Delta S_y = \sqrt{\langle \hat{S}_y^2 \rangle - \langle \hat{S}_y \rangle^2} = \frac{\hbar}{2} \sqrt{1 - \sin^2\theta \sin^2\varphi}.
\]

Assim, o produto das incertezas é:

\[
\Delta S_x \cdot \Delta S_y = \frac{\hbar^2}{4} \sqrt{(1 - \sin^2\theta \cos^2\varphi)(1 - \sin^2\theta \sin^2\varphi)}.
\]

Usando a desigualdade de Heisenberg, devemos ter:

\[
\frac{\hbar^2}{4} \sqrt{(1 - \sin^2\theta \cos^2\varphi)(1 - \sin^2\theta \sin^2\varphi)} \geq \frac{\hbar}{2} \left|\frac{\hbar}{2} \cos\theta \right|.
\]

(c) Verifique que a desigualdade mostrada no item (b) do problema 1 é satisfeita para todos os valores de $\theta$ e $\varphi$.

A desigualdade de Heisenberg deve ser satisfeita para todos os valores de \(\theta\) e \(\varphi\). Precisamos mostrar que:

\[
\sqrt{(1 - \sin^2\theta \cos^2\varphi)(1 - \sin^2\theta \sin^2\varphi)} \geq \cos\theta.
\]

Isso pode ser verificado para diferentes valores de \(\theta\) e \(\varphi\). De fato, a desigualdade é satisfeita para qualquer \(\theta\) e \(\varphi\), uma vez que \(\Delta S_x \cdot \Delta S_y\) é uma expressão não negativa, enquanto o lado direito da desigualdade representa o valor absoluto do componente \(z\) do momento angular.

(d) Quais os estados (especificados pelos valores de $\theta$ e $\varphi$) para os quais a desigualdade acima torna-se uma igualdade?

A desigualdade de Heisenberg se torna uma igualdade quando o estado \(|\chi\rangle\) é um estado coerente de spin, ou seja, quando \( \theta = 0 \) ou \( \theta = \pi \). Nesses casos, \(\Delta S_x\) e \(\Delta S_y\) são minimizadas, e temos:

\[
\Delta S_x \cdot \Delta S_y = \frac{\hbar}{2} \left|\frac{\hbar}{2} \cos\theta \right| = \frac{\hbar^2}{4}.
\]

(e) Quais são os estados para os quais o produto de incertezas $\Delta S_x \Delta S_y$ é máximo? Pertencem ao conjunto de estados obtidos em (c)?

O produto \(\Delta S_x \Delta S_y\) é máximo quando as projeções \( \langle \hat{S}_x \rangle \) e \( \langle \hat{S}_y \rangle \) são pequenas ou zero. Isso ocorre para \(\theta = \frac{\pi}{2}\) (plano equatorial), onde:

\[
\Delta S_x \cdot \Delta S_y = \frac{\hbar^2}{4}.
\]

Estes estados são coerentes com a situação onde \(\cos\theta = 0\), o que não é o caso de \(\theta = 0\) ou \(\theta = \pi\). Portanto, esses estados pertencem ao conjunto de estados com a desigualdade saturada (se tornam iguais).

(f) Quais são os estados para os quais o produto de incertezas $\Delta S_x \Delta S_y$ é mínimo? Pertencem ao conjunto de estados obtidos em (c)?

Para encontrar os estados em que o produto de incertezas \(\Delta S_x \Delta S_y\) é mínimo, vamos analisar as expressões obtidas anteriormente.

A expressão geral para o produto de incertezas é:

\[
\Delta S_x \cdot \Delta S_y = \frac{\hbar^2}{4} \sqrt{(1 - \sin^2\theta \cos^2\varphi)(1 - \sin^2\theta \sin^2\varphi)}.
\]

Para minimizar o produto de incertezas, devemos maximizar as projeções $\langle \hat{S}_x \rangle$ e $\langle \hat{S}_y \rangle$. No caso extremo, isso ocorre quando uma das projeções é máxima (isto é, $\Delta S_x$ ou $\Delta S_y$ é mínima), o que geralmente ocorre para $\varphi = 0 \lor \varphi = \frac{\pi}{2}$.

No entanto, a incerteza será mínima (zero) para o estado em que $\theta = 0 \quad \lor \quad \theta = \pi$, onde o vetor $\hat{n}$ aponta diretamente ao longo do eixo $z$. Nessas situações, $\hat{S}_x \quad \wedge \quad \hat{S}_y$ têm valor esperado zero, o que leva a:

\[
\Delta S_x = \Delta S_y = \frac{\hbar}{2},  
\]
e, portanto 
\[
\Delta S_x \Delta S_y = \frac{\hbar^2}{4}.
\]

No item (c), vimos que a desigualdade de Heisenberg é satisfeita para todos os valores de \(\theta\) e \(\varphi\), mas se torna uma igualdade especificamente para \(\theta = 0 \lor \theta = \pi\). Esses estados correspondem àqueles para os quais o produto de incertezas \(\Delta S_x \Delta S_y\) é mínimo.



\end{enumerate}

\end{document}
